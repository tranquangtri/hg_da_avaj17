\documentclass[12pt]{article}
\usepackage{listing}
\usepackage[utf8]{vietnam}
\usepackage{lmodern}
%\usepackage{xcolor}
\usepackage{array}
\usepackage{listings}
\usepackage{tabularx}
\usepackage{graphicx}
\usepackage{sectsty}
\usepackage{graphics}
\usepackage[table]{xcolor}

\usepackage{hyperref}

\author{Nhóm ABC}
\definecolor{subsection}{RGB}{244, 67, 54}
\definecolor{section}{RGB}{183, 28, 28}
\definecolor{codegray}{rgb}{0.5,0.5,0.5}
\definecolor{codepurple}{rgb}{0.58,0,0.82}
\definecolor{backcolour}{rgb}{0.95,0.95,0.92}

\sectionfont{\color{section}}
\subsectionfont{\color{section}}
\subsubsectionfont{\color{section}}

%.............................
\begin{document}
\title{Phát triển phần mềm nguồn mở\\Đề tài:  Licenses apache}
\maketitle
\tableofcontents
\pagebreak
\section{Thông tin nhóm}
\begin{tabularx}{\textwidth}{|c|c|c|cX|}
\hline
MSSV & Tên & Số điện thoại & Email \\ \hline
1412098 & Trần Văn Duy & 0901378200 & Tvduy1996@gmail.com \\
1412011  & Nguyễn Hoàng Anh &  &  \\
1412111  & Nguyễn Việt Dương &  &  \\
1212525  & Nguyễn Minh Vương &  &  \\
1412578  & Trần Quang Trí & 0913937660 & tranquangtri96@live.com \\
\hline
\end{tabularx}

\section{Nội dung tìm hiểu}

\subsection{Các giấy phép trong nhóm Apache}

\paragraph{Apache 1.0:}
Đây là Giấy phép Apache ban đầu chỉ áp dụng cho các phiên bản Apache cũ hơn (như phiên bản 1.2 của máy chủ Web).

\paragraph{Apache 1.1:}
Phiên bản 1.1 của Giấy phép Apache đã được ASF phê chuẩn năm 2000. Sự thay đổi chính từ giấy phép 1.0 nằm trong mục quảng cáo (phần 3 của giấy phép 1.0), các sản phẩm phát sinh không còn bắt buộc phải kèm lời ghi công trong các thành phần quảng cáo, mà chỉ trong hướng dẫn sử dụng mà thôi.. Các gói riêng lẻ được cấp phép theo phiên bản 1.1 có thể đã sử dụng các từ khác nhau do các yêu cầu khác nhau đối với ghi nhận tác giả hoặc nhận dạng nhãn hiệu, nhưng các điều khoản ràng buộc đều giống nhau. 

\paragraph{Apache 2.0:}
Được ASP bắt đầu sử dụng vào tháng 1 năm 2004. Mục tiêu của giấy phép bao gồm giúp các dự án sử dụng giấy phép nhưng không thuộc ASF dễ dàng sử dụng, cải tiến độ tương thích với phần mềm, cho phép giấy phép được đính vào phần tham khảo thay vì phải ghi trong mỗi tập tin, làm rõ giấy phép cho từng phần đóng góp, và bắt buộc một giấy phép bằng sáng chế cho các đóng góp có khả năng vi phạm bằng sáng chế của một người đóng góp.

\subsection{Phân tích cụ thể từng giấy phép}
\subsubsection{Đối chiếu 10 điều trong OSD}

\begin{tabularx}{\textwidth}{|X|X|}
\hline
Apache 1.0 & OSD \\ 
\hline 
Redistribution and use in source and binary forms, with or without modification \newline
& 
OSD1 (Free redistribution) \newline
OSD3 (Derived works ) \\
\hline


%------Apache 1.1
Apache 1.1 & OSD \\ 
\hline 
Tương tự Apache 1.0 \newline
& 
\\
\hline

%------Apache 2.0
Apache 2.0 & OSD \\ 
\hline 
Redistribution. You may reproduce and distribute copies of the Work or Derivative Works thereof in any medium, with or without modifications, and in Source or Object form
&
OSD1 (Free redistribution) \newline
OSD3 (Derived works )
\\
\hline
\end{tabularx}


\subsubsection{Tính hợp đồng của giấy phép}
\paragraph{Apache 1.0}
\begin{itemize}
\item  Phân phối lại mã nguồn phải giữ lại thông báo bản quyền của giấy phép và một số điều kiện sau: 
\begin{itemize}
\item Phân phối lại dưới dạng nhị phân phải sao chép lại giấy phép trong tài liệu đính kèm bản phân phối. 
\item Tài liệu hướng dẫn người dùng cuối khi phân phối lại nếu có phải bao gồm các xác nhận:
\begin{itemize}
\item “Sản phẩm này bao gồm phần mềm được phát triển bởi Apache software”.
\item Tên "Apache" và "Apache Software Foundation" không được sử dụng để chứng thực hoặc quảng cáo các sản phẩm có nguồn gốc từ phần mềm này mà không có sự cho phép bằng văn bản trước. Để được sự cho phép bằng văn bản, vui lòng liên hệ với apache@apache.org.
\item Sản phẩm có nguồn gốc từ phần mềm này không được gọi là "Apache", cũng không có "Apache" xuất hiện trong tên gọi mà không có sự cho phép bằng văn bản của Apache Software Foundation. 
\end{itemize}
\end{itemize}
\end{itemize}

\paragraph{Apache 1.1}
\begin{itemize}
\item  Tương tự Apache 1.0 và bổ sung Tên "Apache" và "Apache Software Foundation" Không được sử dụng để chứng thực hoặc quảng cáo các sản phẩm có nguồn gốc từ Phần mềm mà không có sự cho phép trước bằng văn bản.
\end{itemize}

\paragraph{Apache 2.0}
\begin{itemize}
\item Cung cấp kèm bản sao của giấy phép khi tái phân phối.
\item Thông báo đã chỉnh sửa khi tái phân phối.
\item Giữ lại mã nguồn, bản quyền, bằng sáng chế, nhãn hiệu, thông báo của mã nguồn gốc, trừ những thông báo không liên quan đến phần nào trong tác phẩm.
\item Nếu tác phẩm có chứa tập tin “Thông Báo” thì khi tái phân phối phải bao gồm một bản sao có thể đọc được các thông tin trong tập tin “Thông Báo”, ngoại trừ những thông báo không liên quan đến tác phẩm.
\item Không cho phép sử dụng với mục đích thương mại tên, nhãn hiệu, hoặc tên sản phẩm của bên cấp phép trừ khi được yêu cầu cho việc sử dụng hợp lý và theo định kì trong việc mô tả nguồn gốc của tác phẩm và sao chép nội dung của tệp “Thông Báo”. 
\end{itemize}

\subsubsection{Một số vấn đề về phần mềm ( tài liệu ) phái sinh}
\paragraph{Apache 1.0}
\begin{itemize}
\item Định nghĩa tác phẩm phái sinh: Tác phẩm phái sinh là tác phẩm dịch từ ngôn ngữ này sang ngôn ngữ khác, tác phẩm phóng tác, cải biên, chuyển thể, biên soạn, chú giải, tuyển chọn.
\item Những quyền và nghĩa vụ của tác giả: 

\begin{itemize}
\item Quyền: Tái phân phối lại và sử dụng trong các dạng mã nguồn và dạng nhị phân, có hoặc không có, Sửa đổi mã nguồn.
\item Nghĩa vụ:

\begin{itemize}
\item Trong mỗi tập tin đã được cấp phép, thông tin về bản quyền và bằng sáng chế trong bản phân phối lại phải được giữ nguyên như ở bản gốc và ở mỗi tệp tin đã được chỉnh sửa phải ghi chú là đã được chỉnh sửa khi nào
\item Không sử dụng tên “Apache Server” và “Apache Group” để xác nhận hoặc quản cáo sản phẩm có nguồn gốc từ phần mềm này mà chưa có văn bản cho phép trước, Để được sự cho phép bằng văn bản, vui lòng liên hệ Apache@apache.org. \item Sản phẩm có nguồn gốc từ phần mềm này không được gọi là "Apache"
\item Phân phối dưới bất kì hình thức nào phải xác nhận "Sản phẩm này bao gồm phần mềm được phát triển bởi Nhóm Apache Để sử dụng trong dự án máy chủ Apache HTTP (http://www.apache.org/). " 
\end{itemize}

\end{itemize}

\item Những giấy phép tiêu biểu tương thích: BSD, MIT,… 
\end{itemize}

\paragraph{Apache 1.1}
\begin{itemize}
\item Định nghĩa tác phẩm phái sinh: Tác phẩm phái sinh là tác phẩm dịch từ ngôn ngữ này sang ngôn ngữ khác, tác phẩm phóng tác, cải biên, chuyển thể, biên soạn, chú giải, tuyển chọn.
\item Những quyền và nghĩa vụ của tác giả: Tương đối giống Apache 1.0 chỉ khác ở quảng cáo sản phẩm không cần xác nhận "Sản phẩm này bao gồm phần mềm được phát triển bởi Nhóm Apache Để sử dụng trong dự án máy chủ Apache HTTP (http://www.apache.org/)."
\item Những giấy phép tiêu biểu tương thích: BSD, MIT…
\end{itemize}

\paragraph{Apache 2.0}
\begin{itemize}
\item Định nghĩa tác phẩm phát sinh: là bất kỳ tác phẩm nào, dù là ở dạng mã nguồn hoặc mẫu đối tượng, dựa trên (hoặc có nguồn gốc từ) tác phẩm và  sửa đổi, chú thích, chỉnh sửa, hoặc các thay đổi khác đại diện cho toàn bộ tác phẩm gốc của tác giả. 
\item Những quyền và nghĩa vụ của tác giả: 

\begin{itemize}
\item Quyền:
\begin{itemize}
\item Tái phân phối lại và sử dụng trong các dạng mã nguồn và dạng nhị phân, có hoặc không có, Sửa đổi mã nguồn. 
\item  Thêm bản tuyên bố bản quyền của riêng bạn vào sửa đổi của bạn và có thể cung cấp các điều khoản và điều kiện cấp phép bổ sung hoặc khác nhau cho việc sử dụng, sao chép hoặc phân phối các sửa đổi của bạn,
\end{itemize}
\item Nghĩa vụ:

\begin{itemize}
\item Cung cấp cho bất kỳ người nhận khác của Tác phẩm hoặc Sản phẩm phái sinh một bản sao của Giấy phép này. 
\item Tạo ra bất kỳ tập tin sửa đổi nào để mang thông báo nổi bật cho biết rằng bạn đã thay đổi các tệp tin 
\item Lưu giữ dưới dạng mã nguồn của bất kỳ Tác phẩm phái sinh nào mà bạn phân phối, tất cả bản quyền, bằng sáng chế, nhãn hiệu và thông báo ghi nhận tác giả từ mã nguồn của tác phẩm, không bao gồm các thông báo không liên quan đến bất kỳ phần nào của tác phẩm phái sinh. 
\item Nếu tác phẩm bao gồm một tập tin văn bản "THÔNG BÁO" trong quá trình phân phối, bất kỳ tác phẩm phái sinh nào mà bạn phân phối phải bao gồm một bản sao của các thông báo thuộc tính có trong tập tin THÔNG BÁO, ngoại trừ các thông báo không liên quan đến bất kỳ phần nào của các tác phẩm phái sinh
\end{itemize}
\item Những giấy phép tiêu biểu tương thích: BSD, MIT… 
\end{itemize}
\end{itemize}

\subsubsection{Một số phần mềm ( tài liệu ) phân phối với giấy phép Apache}
\paragraph{Apache 1.0}
\begin{itemize}
\item ConcertTo,… 
\end{itemize}

\paragraph{Apache 1.1}
\begin{itemize}
\item Apache wave,… 
\end{itemize}

\paragraph{Apache 2.0}
\begin{itemize}
\item  Log4j, IronRuby, Apache TomCat, ConnectBot,… 
\end{itemize}

\subsection{Tóm tắc và so sánh các giấy phép MIT, BSD (3-clause), Apache v1, Apache v2, CCO, CC BY}
\paragraph{MIT}
Là giấy phép cho phép ngắn hạn và đơn giản với điều kiện chỉ yêu cầu bảo quản thông báo bản quyền và giấy phép. Tác phẩm được cấp phép, sửa đổi, và các tác phẩm lớn hơn có thể được phân phối dưới các điều khoản khác nhau và không có mã nguồn.
\paragraph{BSD(3-clause)}
Tương tự như giấy phép MIT  nhưng với điều khoản cấm bên thứ 3 sử dụng tên của dự án hoặc các cộng tác viên để quảng cáo các sản phẩm có nguồn gốc mà không có sự đồng ý bằng văn bản.
\paragraph{CC0}
Giấy phép Creative Commons CC0 sẽ khước từ sở hữu bản quyền đối với bất kỳ tác phẩm nào mà bạn đã tạo ra và dành nó cho toàn bộ miền công cộng. Sử dụng CC0 để loại bỏ hoàn toàn bản quyền và đảm bảo công việc của bạn có phạm vi rộng nhất. Giống như Unlicense và các giấy phép phần mềm điển hình, CC0 từ chối bảo hộ.
\paragraph{CC-BY}
Giấy phép cho phép hầu như bất kỳ việc sử dụng nào được cung cấp thông báo về tín dụng và giấy phép. Thường được sử dụng cho tài sản truyền thông và tài liệu giáo dục. Giấy phép phổ biến nhất cho các ấn bản khoa học mở. Không nên dùng cho phần mềm.

\begin{center}

\includegraphics*[viewport=0 155 460 365, width=\textwidth]{table.eps}
\includegraphics*[viewport=0 0 460 153, width=\textwidth]{table.eps}
\begin{figure}[!h]
\caption{Bảng so sánh một số tính chất cơ bản của các giấy phép.}
\end{figure}
\end{center}


\subsection{Tính permissive trong các giấy phép}
\begin{itemize}
\item Hầu hết các giấy phép đều “permissive” một số quyền cơ bản như:
\begin{itemize}
\item Sử dụng thương mại 
\item Tái phân phối
\item Chỉnh sửa 
\item Sử dụng với mục đích cá nhân
\end{itemize}
\item Một số giấy phép cho phép sử dụng bằng sáng chế trong khi các giấy phép như CC0 và CC - BY giới hạn về quyền này.
\item Dựa vào bảng so sánh trên ta có thể thấy MIT và BSD ( 3-clause ) là những giấy phép “thuần cho phép” hầu như có rất ít điều kiện, các giấy phép này so với phạm vi công cộng thì chỉ giới hạn 1 số vấn đề như trách nhiệm và sự bảo đảm còn lại thì giấy phép này cho phép những quyền tương đối sát với phạm vi công cộng.
\end{itemize}

\subsection{Giấy phép tương thích với GPL}
\begin{itemize}
\item Apache v2 (GPLv3)
\item CC0 
\item MIT 
\item CC BY (v4) ( GPLv3)
\item BSD (3-clause) 
\end{itemize}

\subsection{Giấy phép tương thích và không tương thích với Apache License}
\begin{itemize}
\item Tương thích
\begin{itemize}
\item CC0 
\item MIT 
\item CC BY (v4) ( GPLv3)
\item BSD (3-clause) 
\end{itemize}
\item Không tương thích
\begin{itemize}
\item o	Không có giấy phép không tương thích trong các giấy phép trên.
\end{itemize}
\end{itemize}

\subsection{Vai trò của phần mềm thương mại}
\paragraph{Apache 1.0}
\begin{itemize}
\item Phần mềm có thể được tự do sử dụng, sao chép, sửa đổi, phân phối hoặc bán.
\end{itemize}
\paragraph{Apache 1.1}
\begin{itemize}
\item Phần mềm có thể được tự do sử dụng, sao chép, sửa đổi, phân phối hoặc bán.
\end{itemize}
\paragraph{Apache 2.0}
\begin{itemize}
\item Phần mềm có thể được tự do sử dụng, sao chép, sửa đổi, phân phối hoặc bán.
\item Phần mềm có thể được kết hợp với các sản phẩm khác và phân phối hoặc bán như là gói.
\item Các sản phẩm có nguồn gốc hoặc sửa đổi từ phần mềm được cấp phép có thể được phân phối theo các giấy phép khác.
\end{itemize}

\paragraph{MIT}
\begin{itemize}
\item Phần mềm có thể được tự do Sử dụng, sao chép, sửa đổi, hợp nhất, xuất bản, phân phối, cấp phép thứ cấp hoặc bán
\end{itemize}
\paragraph{BSD}
\begin{itemize}
\item Phần mềm có thể được tự do Sử dụng, sao chép, sửa đổi, hợp nhất, xuất bản, phân phối, cấp phép thứ cấp hoặc bán
\item Tên của chủ sở hữu bản quyền cũng như tên của người đóng góp không được sử dụng để chứng thực hoặc quảng bá các sản phẩm có nguồn gốc từ phần mềm này mà không có sự cho phép bằng văn bản cụ thể.
\end{itemize}

\paragraph{CC0}
\begin{itemize}
\item Phần mềm có thể được tự do Sử dụng, sao chép, sửa đổi, hợp nhất, xuất bản, phân phối, cấp phép thứ cấp hoặc bán 
\end{itemize}
\paragraph{CCBY}
\begin{itemize}
\item Không khuyến khích sử dụng trong phần mềm thương mại.
\end{itemize}

\section{Tham khảo}
\url{https://www.apache.org/licenses/} \\
\url{https://opensource.org/licenses/BSD-3-Clause}\\
\url{https://creativecommons.org/licenses/by/4.0/legalcode}\\
\url{https://creativecommons.org/publicdomain/zero/1.0/legalcode}\\
\url{https://opensource.org/licenses/MIT}\\
\url{https://www.gnu.org/licenses/license-list.html}\\
\url{https://opensource.org/osd-annotated}\\
\url{https://www.apache.org/legal/resolved.html}\\
\url{https://en.wikipedia.org/wiki/Category:Software_using_the_Apache_license}\\
\url{https://choosealicense.com/appendix/}\\
\end{document}
